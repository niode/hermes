\documentclass{article}

\usepackage{amsmath}
\usepackage{amsfonts}

\setlength{\parindent}{0cm}
\begin{document}

\begin{center}
Deliverable 4\\
Nathan Harms, Tristan Schorn
\end{center}

\section{Target Users}

We recruited 3 participants. Our participants were our friends. Our target
audience is ourselves, and those interested in the ideas we are exploring in our
project (procedural generation, etc.). Thus, our friends are good choices.\\

\section{Methods}

We conducted semi-structured interviews. Each participant was asked questions
about the user interface. How well structured is the interface? Is it easy to
tell which parts serve which purpose? How satisfying is the interaction with the
interface? We wrote the answers down.\\

Our questions are open-ended. Scale-based questions would be pointless with such
a small sample. How might we interpret 4/10, for example? What
should we do to get a higher number? We do not know.\\

We then ran through a sample game with the participant. The participant wrote
their commands on paper and we wrote the result, mimicking our prototype. These
sample games are on paper, which we have kept, as reference.\\

We paid attention to which responses caused which reactions from the user. For
example, users often wished for more information about which actions were
available to them.\\

\section{Results}

Our interviewees were frustrated with the lack of direction in the game. They
didn't know what the goal of the game was. Is there a story? It was not clear to
them what to do or why. Our mechanics for procedural generation are not
communicated adequately by merely placing the player in an environment meant for
exploration.\\

The users often tried commands that we had not thought of.\\

The graphical interface is clear to the users, as expected.\\

\section{Analysis of Results}

We must change our design to provide clearer feedback when users use invalid
commands.\\

We ought to find a way to make it clear to the users how the item and world
systems function, rather than leaving it entirely up to exploration.\\

A story would be good too, but difficult to insert into our design. It may be
too open-ended to be feasible.\\

\end{document}
