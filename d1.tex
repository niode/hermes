\documentclass{article}

\usepackage{amsmath}
\usepackage{amsfonts}

\setlength{\parindent}{0cm}
\begin{document}

\begin{center}
"Deliverable 1"\\
Nathan Harms, Tristan Schorn
\end{center}

\section{Description}
We intend to create a text-based adventure (or ``Interactive Fiction") game
targeted to the Linux command-line. We will focus on procedurally generated
content rather than a scripted story.

\section{Features}
\begin{enumerate}
  \item A parser that allows the user to interact with the game through commands
    that resemble natural language.
  \item A procedural world generator which produces structured worlds that make
    sense.
  \item A procedural action generator which determines which actions a player
    can take based on context, e.g. which actions can be performed on an object.
  \item A simulation which determines the consequences of actions on objects.
  \item A system which generates descriptions of the above actions to inform the
    user of the effect they are having.
\end{enumerate}

Beyond these, we may attempt more features if time permits. These may include:
\begin{enumerate}
  \item Generation of image-based graphics.
  \item A narrative.
\end{enumerate}

\section{Competition}
Text-based adventures and ``interactive fiction" are not highly competitive.
However, there are many hobbyists who produce work in this area, and there are
some technologies that facilitate this, such as the Gargoyle player. There is
also academic research being done in the area, as it is a simple testbed for
ideas in natural language processing, AI, and computer creativity.

The Gargoyle player (ccxvii.net/gargoyle) is an interpreter for games which
focuses on having a well-designed interface and good typography, as well as
being an all-in-one interpreter for various formats of games. Our system is not
an interpreter and will focus on content-generation as opposed to interface
design.

A system called Curveship (curveship.com) has been created by Nick Montfort at
MIT and has similar aims as our system. Our game will be a standalone game and
not a system for developing games, however.

\end{document}
